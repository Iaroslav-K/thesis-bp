Všichni se během svého života setkávají s člověkem, se kterým se stávají nerozluční. Časem se přichází svatba a dětí. I když myslíme, že láska je navěky, pořad se lidé rozvádí. Většinou manžele po rozvodu chtějí co nejméně komunikovat mezi sebou. Konflikt se může vzniknout kvůli čemu-koliv. Manžele můžou nerespektovat jeden druhého nebo dokonce i dělat něco naschvál druhému. Rozvadění je těžké pro obou z manželů, ale pro dětí je tohle období daleko horší. Dětí se často stávají v teto situaci centrem konfliktu.  


Jedním ze způsobů jak odstínit dítě od konfliktu samotného je zabezpečit komunikaci manželů pomoci aplikaci,  funkcionalita které by měla pokrývat nejdůležitější aspekty které mají vliv na dítěte, což je zprava pečovatelských dnů, zprava alimentů a zprava nákupu pro dítěte. Tyto věcí brání rodičům využit dítěte jako zbraň v konfliktech mezi sebou.

Toto téma jsem vybral, protože jsem chtěl pracovat nad zajímavým reálným projektem, který v budoucnu budou lidi používat. V rámci zadání mám pracovat s moderním programovacím jazykem Kotlin. Mám rád ideologii tohoto jazyku, proto je mi příjemné pracovat nad implementací projektu.


Projekt se skládá z mobilní aplikaci, kterou souběžně řeší v rámci bakalářské práci Martin Beran, a Serverové častí, kterou řeším v rámci teto bakalářské práci. V analýze vycházím z výsledku předmětů BI-SP1 a BI-SP2. Zúčastnil jsem se obou předmětu a vystupoval jsem hlavně jako backend programátor. V rámci druhé častí softwarového projektu také jsem byl vedoucím backendového týmu.


Tato práce se skládá z pěti kapitol. První kapitola je věnovaná rešerše, Druha kapitola je určena analýze. Třetí kapitola – návrhu. Čtvrtá kapitola – Implementaci. A poslední, pátá kapitola, je věnovaná testovaní serveru.