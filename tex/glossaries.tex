\chapter{Slovník pojmů}
\begin{description}[leftmargin=12em,style=nextline] % TODO dodelat vsechny, upravit do spravneho poradi 
    \item[Framework] Framework je softwarová struktura orientovaná na vyřešení problému nebo jeho zjednodušení při procesu vývoje softwarových projektu
	\item[Backend] Část aplikace, která se stará o data a jejích spravování. Pro interakci s backendem klient většinou potřebuje přistup do frontendové části aplikace
    \item[Frontend] Prezentáční vrstva aplikace se kterou interaguje klient. 
	\item[DSL jazyk] Programovací jazyk nebo použití obecného programovacího jazyku, vytvořený za účelem vyřešení konkretní problémové domény
	\item[Interní DSL jazyk] DSL jazyk využívající obecný programovací jazyk
	\item[Unit testy] Testy, zaměřené na ověření správnosti fungování samostatně testovatelné části programu
	\item[Integrační testy] Testy, zaměřené na ověření správné komunikace mezi komponentami
	\item[Inversion of Controle] Princip, za kterého kontrola nad vytvořením a provázáním tříd vlastní framework. 
	\item[Git]  Git je distribuovaný systém řízení verzí
	\item[Commit]  Commit je proces při kterém se uloží všechny udělané v rámci systému řízení verzí změny a zařadí se do historie změn
	\item[GitLab] GitLab je webový platformou pro vývoj softwaru pomocí systému řízení verzí Git
	\item[Diagram Užití] Diagram Užití popisuje chování systému z vnějšího pohledu
	\item[Diagram aktivit] Diagram aktivit zobrazuje jak objekty spolupracují
	\item[Wireframe] Wireframy je grafické zobrazení hlavních prvků frontendové častí aplikace
	\item[Doménový model] Doménový model je náčrtem základních entit systému a vztahů mezi nimi
	\item[CLOC] Nástroj poskytující možnost spočítat počet řádek kódu v dáne složce. Nástroj podporuje velký počet jazyku programování. Výsledek obsahuje počet řádek kódu oddělený od komentářů a prázdných řádků.
	\item[mapování] Proces přidání konkrétnímu controlleru adresy pomocí anotací frameworku Spring, která se zadává jako URI při odesílaní požadavků na Server
	\item[Kontext aplikace] Pokročilý kontejner, který funguje podobně \textit{BeanFactory}. Načítá definice beanů, provazuje je a vydává v případe nutnosti
\end{description}