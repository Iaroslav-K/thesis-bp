\chapter{Slovník pojmů}
\begin{description}[leftmargin=12em,style=nextline] 
    \item[Access token] Token, který klient používá pro komunikaci se serverem.
	\item[Backend] Část aplikace, která se stará o data a jejich spravování. Pro interakci s backendem klient většinou potřebuje přistup do frontendové části aplikace.
	\item[CLOC] Nástroj poskytující možnost spočítat počet řádků kódu v dané složce. Nástroj podporuje velký počet jazyků programování. Výsledek obsahuje počet řádků kódu oddělený od komentářů a prázdných řádků.
	\item[Commit]  Proces, při kterém se uloží všechny změny provedené v rámci systému řízení verzí a zařadí se do~historie změn.
    \item[Datová vrstva] Vrstva aplikace, která komunikuje s databází.
	\item[DSL jazyk] Programovací jazyk (nebo použití obecného programovacího jazyka) vytvořený za účelem vyřešení konkrétní problémové domény.
	\item[Diagram aktivit] Diagram aktivit zobrazuje, jak objekty spolupracují.
	\item[Diagram Užití] Diagram Užití popisuje chování systému z vnějšího pohledu.
	\item[Doménový model] Náčrt základních entit systému a vztahů mezi nimi.
	\item[Drátový model] Grafické zobrazení hlavních prvků frontendové části aplikace.
	\item[Git] Distribuovaný systém řízení verzí.
	\item[GitHub] Webová platforma pro vývoj softwaru pomocí systému řízení verzí Git.
	\item[GitLab] Webová platforma pro vývoj softwaru pomocí systému řízení verzí Git.
    \item[Framework] Softwarová struktura orientovaná na vyřešení problému nebo jeho zjednodušení při procesu vývoje softwarových projektu.
    \item[Frontend] Prezentační vrstva aplikace se kterou interaguje klient.
	\item[Integrační testy] Testy zaměřené na ověření správné komunikace mezi komponentami.
	\item[Interní DSL jazyk] DSL jazyk využívající obecný programovací jazyk.
	\item[Inversion of Controle] Princip, podle kterého kontrolu nad vytvořením a provázáním tříd vlastní framework. 
	\item[Kontext aplikace] Pokročilý kontejner, který funguje podobně. \textit{BeanFactory} Načítá definice beanů, provazuje je a vydává v případě nutnosti.
	\item[Mapování] Proces přidání adresy konkrétnímu controlleru pomocí anotací frameworku Spring, která se zadává jako URI při odesílaní požadavků na Server.
	\item[Refresh token] Token, který klient používá pro obnovení access tokenu po vypršení jeho platnosti.
	\item[Token] Alfanumerické heslo.
	\item[Unit testy] Testy zaměřené na ověření správnosti fungování samostatně testovatelné části programu.
\end{description}