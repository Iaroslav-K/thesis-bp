\chapter{Testování}\label{testovani}
% Nedílnou součástí vývoje softwaru je testování. Testování bylo rozděleno do dvou vrstev. Základní vrstvou jsou unit testy. Pro tvorbu těchto testů byl použit framework JUnit 5. Druhou vrstvou jsou integrační testy, které jsou implementovány pomoci funkcionality, kterou poskytuje framework Spring.

\section{Tagy}\label{testovani:tagy}
    % Během vývoje velkých projektů zpouštění testů je významným problémem pro programátory. 
    % Framework JUnit 5 poskytuje možnost označovat metody a třídy pomocí tagu.
\section{Zobrazování testů}\label{testovani:zobrazovani}
\section{Integrační testy}\label{testovani:intergacni}
\section{Unit testy}\label{testovani:unit}
\section{Pokrytí kódu testy}\label{testovani:pokryti}
    % TODO smazat subsection pokud nebude pridan jeste jeden
    \subsection{JoCoCo}
    \cite{JoCoCo}
    % vestaveny junit engine v IntelifIDEA prestal fingovat po dosazeni velkeho poctu testu
    % \subsection{IntelliJ IDEA code coverage runner}
    % \cite{IntelliJ IDEA code coverage runner}