\chapter{Testování}\label{testovani}
% Nedílnou součástí vývoje softwaru je testování. Testování bylo rozděleno do dvou vrstev. Základní vrstvou jsou unit testy. Pro tvorbu těchto testů byl použit framework JUnit 5. Druhou vrstvou jsou integrační testy, které jsou implementovány pomoci funkcionality, kterou poskytuje framework Spring.
V teto kapitole bude popsán proces testování, doplňková funkcionalita, zjednodušující proces testování a také popis pokrytí kódů testy.
\section{Tagy}\label{testovani:tagy}
    TODO popis tagu a jejich implementace
    % Během vývoje velkých projektů zpouštění testů je významným problémem pro programátory. 
    % Framework JUnit 5 poskytuje možnost označovat metody a třídy pomocí tagu.
\section{Zobrazování testů}\label{testovani:zobrazovani}
    TODO popis CamelCaseGeneratoru a displayname
\section{Integrační testy}\label{testovani:intergacni}
    TODO podrobny popis
\section{Unit testy}\label{testovani:unit}
    TODO podrobny popis
\section{Pokrytí kódu testy}\label{testovani:pokryti}
    TODO V teto sekci je uveden popis provedéní analýzy pokrýtí kódu testy.
    % TODO smazat subsection pokud nebude pridan jeste jeden
    \subsection{JoCoCo}
    TODO pokryti kodu testy podle jacoco
    \cite{JoCoCo}
    % vestaveny junit engine v IntelifIDEA prestal fingovat po dosazeni velkeho poctu testu
    % \subsection{IntelliJ IDEA code coverage runner}
    % \cite{IntelliJ IDEA code coverage runner}