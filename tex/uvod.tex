
Rozvodové řízení je komplikovaný a nepříjemný proces. V~současné době existují různé postupy, které by měly tento proces usnadnit: smluvený nesporný rozvod, předmanželská smlouva a další. Čím méně komunikace vyžaduje rozvodové řízení, tím snadnější je tento proces pro manžele. Ale v~případě, že pár má děti, komunikace mezi rodiči musí pokračovat. 

Jedním ze způsobů, jak ochránit děti před konfliktem samotným, je zabezpečit komunikaci rodičů pomoci aplikace. Funkce aplikace by měly pokrývat nejdůležitější aspekty, které mají vliv na děti, což je správa pečovatelských dnů, alimentů či požadavků dítěte. Takový způsob komunikace rodičů by měl, nejenom odstínit děti od konfliktu, ale i zabránit rodičům využít je v~konfliktech mezi sebou.

Výše zmíněná aplikace se podle požadavků zákazníka skládá z~Android aplikace, kterou současně řeší kolega Martin Beran, a serverového backendu, který je předmětem této bakalářské práce. Předešlý stav projektu je výsledkem předmětu BI-SP1 a BI-SP2, vyučovaných na FIT ČVUT v~Praze. Zmíněné předměty jsou zaměřené na studium pomocí praktického vyzkoušení analýzy, návrhu a realizace rozsáhlejšího softwarového systému. V~rámci těchto předmětů byly navrhnuty backendové a frontendové části aplikace a byla provedená částečná implementace. 
Výsledná implementace serverového backendu bude poskytovat RESTové služby pro Android aplikaci a spravovat procesy, které jsou nezávislé na frontendové části aplikace. Pro implementaci byl zvolen programovací jazyk Kotlin a framework Spring. 

Autor této práce se zúčastnil zmíněných předmětů a je seznámený s předešlým návrhem aplikace. Během předmětu BI-SP2 pracoval na backendu aplikace a také vystupoval v~roli vedoucího backendového týmu. Během procesu implementace projektu obdržel nabídku pokračovat na backendu dané aplikace v~rámci bakalářské práce. Práce na projektu v~rámci bakalářské práce začala ihned po dokončení předmětu BI-SP2, který byl absolvován autorem v~zimním semestru akademického roku 2019/2020.

Cíle této bakalářské práce jsou navrhnout vhodné úpravy a následně implementovat serverový backend aplikace na základě existujícího návrhu a částí implementace v souladu se současným stavem implementace Android aplikace. Po dosažení funkčního výsledku zhodnotit použitelnost a navrhnout budoucí kroky.

Práce se skládá z~pěti kapitol. První kapitola je věnována popisu používaných nástrojů. Druhá kapitola se zabývá analýzou současného návrhu a existujících částí implementace. Třetí kapitola představuje návrh změn a následné implementace navržených změn a ostatních funkcí. Čtvrtá kapitola je věnovaná testování implementovaného softwaru. V~páte kapitole bude zhodnocena výsledná implementace a navrženy budoucí kroky.






