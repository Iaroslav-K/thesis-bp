% Všichni se během svého života setkávají s člověkem, se kterým se stávají nerozluční. Časem se přichází svatba a dětí. I když myslíme, že láska je navěky, pořad se lidé rozvádí. Většinou manžele po rozvodu chtějí co nejméně komunikovat mezi sebou. Konflikt se může vzniknout kvůli čemu-koliv. Manžele můžou nerespektovat jeden druhého nebo dokonce i dělat něco naschvál druhému. Rozvadění je těžké pro obou z manželů, ale pro dětí je tohle období daleko horší. Dětí se často stávají v teto situaci centrem konfliktu.


% Jedním ze způsobů jak odstínit dítě od konfliktu samotného je zabezpečit komunikaci manželů pomoci aplikaci,  funkcionalita které by měla pokrývat nejdůležitější aspekty které mají vliv na dítěte, což je zprava pečovatelských dnů, zprava alimentů a zprava nákupu pro dítěte. Tyto věcí brání rodičům využit dítěte jako zbraň v konfliktech mezi sebou.

% Toto téma jsem vybral, protože jsem chtěl pracovat nad zajímavým reálným projektem, který v budoucnu budou lidi používat. V rámci zadání mám pracovat s moderním programovacím jazykem Kotlin. Mám rád ideologii tohoto jazyku, proto je mi příjemné pracovat nad implementací projektu.


% Projekt se skládá z mobilní aplikaci, kterou souběžně řeší v rámci bakalářské práci Martin Beran, a Serverové častí, kterou řeším v rámci teto bakalářské práci. V analýze vycházím z výsledku předmětů BI-SP1 a BI-SP2. Zúčastnil jsem se obou předmětu a vystupoval jsem hlavně jako backend programátor. V rámci druhé častí softwarového projektu také jsem byl vedoucím backendového týmu.


% Tato práce se skládá z pěti kapitol. První kapitola je věnovaná rešerše, Druha kapitola je určena analýze. Třetí kapitola – návrhu. Čtvrtá kapitola – Implementaci. A poslední, pátá kapitola, je věnovaná testovaní serveru.

I když si spousta z nás namlouvá, že láska je navěky, pořad se lidé rozvádí. Tenhle proces není příjemný, ale zároveň i nemusí být katastrofou. Rozvádějící se manžele mohou volit smluvený nesporný rozvod, i když je to složitě. Situace se stává mnohem komplikovanější v případě, že manžele mají děti, kteří se stávají centrem všem konfliktu. Libovolné řešení, které se týká dítěte, se stává probléme, protože v teto situace manžele už nemůžou se mírně dohodnout na nějakém řešení. Takovým problémem může stát, jak nakoupení bot, tak i výlet do dětského tábora. Libovolná komunikace mezi partnery může vyvolat konflikt, který dětí přenáší obtížně.

Jedním ze způsobů, jak odstínit dětí od konfliktu samotného, je zabezpečit komunikaci manželů pomoci aplikace. Funkcionalita by měla pokrývat nejdůležitější aspekty, které mají vliv na dítěte, což je správa pečovatelských dnů / alimentů / zprava požadavků dítěte. Takový způsob komunikace manželů by měl, nejenom odstínit dětí od konfliktu, ale i zabránit manželům využit je v konfliktech mezi sebou.

Projekt se skládá z Android aplikace, kterou současně řeší kolega - Martin Beran - a serverového backendu, který předmětem teto bakalářské práce. Současný stav projektu je výsledkem předmětu BI-SP1 a BI-SP2, vyučovaných na FIT ČVUT. V rámci těchto předmětů byl udělán návrh backendové a frontendové častí aplikace a také byla provedená částečná implementace. Server by měl poskytovat RESTove služby pro Android aplikaci a spravovat procesy, které jsou nezávisle na frontendová části aplikace. Pro implementaci byl zvolen programovací jazyk Kotlin a framework Spring.

Autor teto práce se zúčastnil zmíněných předmětu a je seznámený s problematikou rozvodů, současným návrhem aplikace a používanými technologií. Během předmětu BI-SP2 autor práce pracoval na backendu aplikace a také vystupoval v roli vedoucího backendového týmu. Po obdržení nabídky pokračování na backendu dáne aplikace v rámci bakalářské práce, souhlas autora byl okamžitý a jednoznačný.

Cílem práce navrhnout vhodné úpravy a následné implementovat serverový backend aplikace na základě existujícího návrhu a častí implementace. Po dosažení funkčního výsledku zhodnotit použitelnost a navrhnout budoucí kroky.

Práce se skládá ze šestí kapitol. První kapitola je rešerší používaných nástrojů. Druha kapitola je věnovaná analýze současného návrhu a existujících části implementace. Třetí kapitola se zabývá návrhem změn a následné implementace. Čtvrtá kapitola je věnovaná testování implementovaného softwaru. Páta kapitola je věnovaná zhodnocení výsledné implementace a navrženi budoucích kroků. Poslední, šestá, kapitola je věnovaná závěru.






