% Všichni se během svého života setkávají s člověkem, se kterým se stávají nerozluční. Časem se přichází svatba a dětí. I když myslíme, že láska je navěky, pořad se lidé rozvádí. Většinou manžele po rozvodu chtějí co nejméně komunikovat mezi sebou. Konflikt se může vzniknout kvůli čemu-koliv. Manžele můžou nerespektovat jeden druhého nebo dokonce i dělat něco naschvál druhému. Rozvadění je těžké pro obou z manželů, ale pro dětí je tohle období daleko horší. Dětí se často stávají v této situaci centrem konfliktu.


% Jedním ze způsobů jak odstínit dítě od konfliktu samotného je zabezpečit komunikaci manželů pomoci aplikaci,  funkcionalita které by měla pokrývat nejdůležitější aspekty které mají vliv na dítěte, což je zprava pečovatelských dnů, zprava alimentů a zprava nákupu pro dítěte. Tyto věcí brání rodičům využit dítěte jako zbraň v konfliktech mezi sebou.

% Toto téma jsem vybral, protože jsem chtěl pracovat nad zajímavým reálným projektem, který v budoucnu budou lidi používat. V rámci zadání mám pracovat s moderním programovacím jazykem Kotlin. Mám rád ideologii tohoto jazyku, proto je mi příjemné pracovat nad implementací projektu.


% Projekt se skládá z mobilní aplikaci, kterou souběžně řeší v rámci bakalářské práci Martin Beran, a Serverové častí, kterou řeším v rámci této bakalářské práci. V analýze vycházím z výsledku předmětů BI-SP1 a BI-SP2. Zúčastnil jsem se obou předmětu a vystupoval jsem hlavně jako backend programátor. V rámci druhé častí softwarového projektu také jsem byl vedoucím backendového týmu.


% Tato práce se skládá z pěti kapitol. První kapitola je věnovaná rešerše, Druha kapitola je určena analýze. Třetí kapitola – návrhu. Čtvrtá kapitola – Implementaci. A poslední, pátá kapitola, je věnovaná testovaní serveru.

% I když si spousta z nás namlouvá, že láska je navěky, pořad se lidé rozvádí. Tenhle proces není příjemný, ale zároveň i nemusí být katastrofou. Rozvádějící se manžele mohou volit smluvený nesporný rozvod, i když je to složité. Situace se stává mnohem komplikovanější v případě, že manžele mají děti, kteří se stávají centrem všech konfliktu. Libovolné řešení, které se týká dítěte, se stává probléme, protože v této situace manžele už nemůžou se mírně dohodnout na nějakém řešení. Takovým problémem může stát, jak nakoupení bot, tak i výlet do dětského tábora. Libovolná komunikace mezi partnery může vyvolat konflikt, který dětí přenáší obtížně.

Rozvodové řízení je komplikovaný a nepříjemný proces. V~současné době existují různé postupy, které by měly tento proces usnadnit: smluvený nesporný rozvod, předmanželská smlouva a další. Čím méně komunikace vyžaduje rozvodové řízení, tím snadnější je tento proces pro manžele. Ale v~případě, že pár má děti, komunikace mezi rodiči musí pokračovat. 

Jedním ze způsobů, jak ochránit děti před konfliktem samotným, je zabezpečit komunikaci rodičů pomoci aplikace. Funkce aplikace by měly pokrývat nejdůležitější aspekty, které mají vliv na děti, což je správa pečovatelských dnů, alimentů či požadavků dítěte. Takový způsob komunikace rodičů by měl, nejenom odstínit děti od konfliktu, ale i zabránit rodičům využít je v~konfliktech mezi sebou.

Výše zmíněná aplikace se podle požadavků zákazníka skládá z~Android aplikace, kterou současně řeší kolega Martin Beran, a serverového backendu, který je předmětem této bakalářské práce. Současný stav projektu je výsledkem předmětu BI-SP1 a BI-SP2, vyučovaných na FIT ČVUT v~Praze. Zmíněné předměty jsou zaměřené na studium pomocí praktického vyzkoušení analýzy, návrhu a realizace rozsáhlejšího softwarového systému. V~rámci těchto předmětů byly navrhnuty backendové a frontendové části aplikace a byla provedená částečná implementace. 
Výsledná implementace serverového backendu bude poskytovat RESTové služby pro Android aplikace a spravovat procesy, které jsou nezávislé na frontendové části aplikace. Pro implementaci byl zvolen programovací jazyk Kotlin a framework Spring. 

Autor této práce se zúčastnil zmíněných předmětů a je seznámený se současným návrhem aplikace. Během předmětu BI-SP2 pracoval na backendu aplikace a také vystupoval v~roli vedoucího backendového týmu. Během procesu implementace projektu obdržel nabídku pokračovat na backendu dané aplikace v~rámci bakalářské práce. Práce na projektu v~rámci bakalářské práce začala ihned po dokončení předmětu BI-SP2, který byl absolvován autorem v~akademickém roce 2019/2020.

Cíle této bakalářské práce jsou navrhnout vhodné úpravy a následně implementovat serverový backend aplikace na základě existujícího návrhu a částí implementace po dosažení funkčního výsledku zhodnotit použitelnost a navrhnout budoucí kroky.

Práce se skládá z~pěti kapitol. První kapitola je věnována popisu používaných nástrojů. Druhá kapitola se zabývá analýzou současného návrhu a existujících částí implementace. Třetí kapitola představuje návrh změn a následné implementace navržených změn a ostatních funkcí. Čtvrtá kapitola je věnovaná testování implementovaného softwaru. V~páte kapitole bude zhodnocena výsledná implementace a navrženy budoucí kroky.






