%space 1
%space 2 I hate this box on the top right corner
Projekt byl rozdělen do dvou samostatných častí, které se vyvíjely paralelně. První častí je Android aplikace. Druhou částí, která je předmětem této bakalářské práci, je serverový backend, poskytující RESTové služby pro Android aplikace.

Cílem teto práce bylo navržení vhodných úprav a následná implementace existujícího návrhu a fragmentů implementace. A také, zhodnocení použitelností výsledné implementace a navržení vhodných budoucích kroků pro pokračování vývoje serveru. Při implementaci byl také uvažován současný stav, souběžně se vyvíjející, frontendové částí aplikace.

I když autor této práci se zúčastnil předmětů, během kterých byl udělán předchozí návrh aplikace a fragmenty implementace, byla potřeba přezkoumat návrh programu za účelem eliminování nalezených během implementace frontendové a backendové častí nedostatků a navrhnout lepší řešení. Předchozí implementace programu pokrývala jenom malou část aplikace, proto, za účelem lepšího výsledku, skoro celá již existující implementace byla přepsána. Byl opraven doménový model, provedeny změny použité třívrstvé architektury, rozšířené API, rozšířená dokumentace API, přidán proces registraci a přihlašování uživatele a také implementována bezpečnost aplikace. Při implementaci byly také uvazovány požadavky kolegy, který souběžně vyvíjí frontendovou část aplikace.

% Práce se začíná analýzou existujícího návrhu a fragmentů implementace. I když jsem zúčastnil předmětů, které se zabývali těmto návrhem a implementací, potřeboval jsem přezkoumat celý návrh za účelem eliminování, nalezených během implementace frontendové a backendové častí, nedostatků a navržení lepších řešení.
% Potom jsem návrh novou verzí řešení, která obsahuje úpravy podle požadavků frontendové části aplikace a zároveň obsahuje navržené mnou úpravy. Implementace navrženého řešení byla nejsložitější častí teto bakalářské práci, protože jsem neměl zkušeností ve vývoje projektů porovnatelného rozsahu a potřeboval jsem naučit mnoha novým věcem.
Důležitou částí teto práce bylo testování, které v rozsahu napsaného kódu je porovnatelnou s implementací serveru samotného. Testy byly rozděleny do dvou typu: unit testy a integrační testy. Unit testy testují samostatně testovatelné funkci programu. Integrační testy využívají za běhu celý kontext aplikace a testují jestli jednotlivé řadiče fungují správné. 
% \begin{figure}\centering
% 	   \includegraphics[angle=-90, width=0.5\textwidth]{pdfs/CodeAmountTests2}
% 	   \caption[Analýza kódu testů]{Počet napsáných řádek kódu testů}\label{image:code-count-test}
% \end{figure}
% \begin{figure}\centering
% 	   \includegraphics[angle=-90, width=0.5\textwidth]{pdfs/CodeAmountImpl2}
% 	   \caption[Analýza kódu implementace]{Počet napsáných řádek kódu za účelem implementace funkcionality}\label{image:code-count-main}
% \end{figure}

Výsledkem této bakalářské práce je funkční aplikace splňující všechny požadavky frontendové častí aplikace. Ale, jak Android aplikace, tak i serverový backend ještě nejsou ve svém finálním stavu. V rámci této práce byl zhodnocen současný stav a nazřeny vhodné budoucí kroky pro pokračování vývoje.
% Následující kapitola se věnuje zhodnocení použitelností výsledné implementace a navržení budoucích kroků. Výsledek ještě není připravený pro produkční prostředí a ještě ho očekává dostatečný počet vylepšení. V rámci teto bakalářské práci jsem implementoval základ backendu, který má skoro celou potřebnou funkcionalitu, kromě některých věcí, které byly objevený už během pečlivého zanoření do implementace a za
% řazené do budoucích kroků.