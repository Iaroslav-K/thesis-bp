\chapter{Rešeře}\label{resere}
\section{Kotlin}\label{resere:kotlin}
    Kotlin je relativně nový jazyk pro JVM. Poprvé tento jazyk byl představen společnosti v 2011 roku. První stabilní verze byla představena v únoru roku 2016. Ale už v květnu roku 2017 Kotlin se stal oficiálním jazykem pro Android.
    
    Kotlin byl vytvořen jako alternativa jazyku Java a řeší některé jí problémy. Například, Kotlin řeší problém použití null, také známý jako \textit{The Billion Dollar Mistake}\cite{theBDM}, a spojené s ní problémy. Java samotna nemá podporu pro \textit{not-null} proměnné, ale Kotlin takovou podporu má, a to v podobě oddělení \textit{nullable} typu pomocí ? operátoru.
    
    Kotlin je odlišný od Javy i syntaxí. Například není potřeba psát středník pro dokončeni příkazu, ale je vyžadován pouze v případe, že chcete oddělit několik příkazů na jedné řádce. Také byly odstraněny spousta klíčových slov. Například pro deklarací \textit{public final} proměnné je potřeba použit klíčově slovo \textit{val}. Důležitým rozdílem je, že při vytvoření třídy Kotlin umí vygenerovat metody \textit{get} a popřípadě i \textit{set} a umožňuje zadávat defaultní hodnoty v konstruktoru. Také Kotlin zavádí \textit{data class}, který navíc od obyčejné třídy umí vygenerovat metod \textit{toString}, který převádí třídu na typ \textit{String} v čítelné podobě \cite{Priklad vygenerovane tridy}, a metody \textit{equals} a \textit{hashCode}. Celý seznam věcí, které Kotlin má navíc od Javy nebo má jinou implementaci, což není předmětem teto práci, nemá smysl. Proto 
    
    
    Kotlin je Kotlin podporuje \textit{type-safe builder}
\section{Spring}\label{resere:j2ee}
    
\section{Testování}\label{resere:testovani}
    \subsection{Spring}
    TODO popis nastroju pro testovani
    \subsection{JUint 5}
    TODO popis nastroju pro testovani
    \subsection{JaCoCo}\label{resere:testovani:jacoco}
    TODO nastroj pro analyzu pokryti kodu testy \cite{JoCoCo}
    \subsection{IntelliJ IDEA}\label{resere:testovani:intellij-idea}
    TODO nastroj pro analyzu pokryti kodu testy \cite{intellij-idea-code-coverage}

\section{Dokumentace}\label{resere:dokumentace}
    Swagger
\section{Databáze}\label{resere:databaze}
    H2, PostgresSQL
\section{Buildováci sstém}\label{resere:build}
    TODO Gradle...
